\RequirePackage[l2tabu, orthodox]{nag}
\documentclass[letterpaper]{amsart}

\usepackage[utf8]{inputenc}
%\usepackage[T2A]{fontenc}

% MATH
\usepackage{amsmath}
\usepackage{amsthm}
\usepackage{amsfonts}
\usepackage{amssymb}
\usepackage{mathtools}

\usepackage[
style=alphabetic
]{biblatex}
\addbibresource{bibliography.bib}

\usepackage{hyperref}
\hypersetup{
    colorlinks,
    citecolor=blue,
    filecolor=green,
    linkcolor=darkgray,
    urlcolor=gray
}

%\usepackage[left=2cm,right=2cm,top=2cm,bottom=2cm]{geometry}
\author{Boris Bilich}
\address{Department of Mathematics, National Research University "Higher School of Economics",
Moscow, Russian Federation}
\email{bibilich@edu.hse.ru}
\title{Taylor spectrum for modules over Lie algebra}

\usepackage{todonotes}

\newtheorem{theorem}{Theorem}
\newtheorem{lemma}[theorem]{Lemma}
\newtheorem{proposal}{Proposal}[theorem]
\newtheorem{corollary}{Corollary}[theorem]
\newtheorem*{remark}{Remark}
\newtheorem{definition}{Definition}
\newtheorem{example}{Example}
\renewcommand{\qedsymbol}{$\blacksquare$}

\newcommand{\cmpl}{\mathbb{C}}
\newcommand{\lieg}{\mathfrak{g}}
\newcommand{\lieh}{\mathfrak{h}}
\newcommand{\lies}{\mathfrak{s}}
\newcommand{\lien}{\mathfrak{n}}
\newcommand{\Ext}{\operatorname{Ext}}
\newcommand{\Hom}{\operatorname{Hom}}
\newcommand{\ad}{\operatorname{ad}}
\newcommand{\Tor}{\mathrm{Tor}}
\newcommand{\ZZ}{\mathbb{Z}}
\newcommand{\RR}{\mathbb{R}}
\newcommand{\CC}{\mathbb{C}}

\newcommand{\lmod}[1]{\operatorname{\mathbf{#1-mod}}}
\newcommand{\rmod}[1]{\operatorname{\mathbf{mod-#1}}}

\setcounter{tocdepth}{2}
\setcounter{secnumdepth}{2}


\begin{document}
\maketitle

\begin{abstract}
    In the paper we generalize the notion of Taylor spectrum to modules over an arbitary Lie
    algebra and study it for finite-dimensional modules. We show, that in case of nilpotent and
    semisimple Lie algebras, the spectrum can be described as the set of simple submodules. We also
    show, that this result does not hold for solvable Lie algebras and give the precise description
    of the spectrum in case of the Borel subalgebra of semisimple Lie algebra.
\end{abstract}

\tableofcontents
\section{Introduction}%
\label{sec:introduction}
In 1970, J. Taylor\cite{Taylor1970} introduced the notion of the joint spectrum for a finite tuple
of commuting operators $T = (T_1,...,T_n)$, acting on a Banach space $V$. It was defined as a
subset of $\CC^n$ (where $n$ is number of operators), which consists of such elements
$(\lambda_1,...,\lambda_n)$, that the Koszul complex for algebra $\mathcal{B}(V)$ of bounded
operators and elements $T_1 - \lambda_1,...,T_n - \lambda_n$ is not exact. For one operator, the
Taylor spectrum coincides with the classical one. In the same year, Taylor estabilished the
existence of the holomorphic functional calculus in the neighbourhood of the spectrum
\cite{Taylor1970b}. Two years later, he proposed a framework for noncommutative functional
calculus\cite{Taylor1972}, but the notion of spectrum for non-commuting tuples of operators was not
yet developed. 

The step in this direction was made by A.S. Fainshtein in his work \cite{Fainshtein}, where he
generalized the Taylor spectrum to tuples of operators, generating nilpotent Lie algebra $\lieg$.
This definiton was soon improved by E. Boasso and A. Larotonda \cite{boasso} to fit also for solvable Lie
algebras. Let us recall the definiton, given by them.
\begin{definition}
    Let $\lieg$ be a solvable Lie algebra, and $V$ be a right $\lieg$-module. The Taylor spectrum
    of $V$ is the subset of the space of characters $(\lieg/[\lieg,\lieg])^*$ of the form
    \begin{equation} \label{e:boassolarotonda}
        \sigma^{BL}_\lieg(V) = \{ \lambda \in (\lieg/[\lieg,\lieg])^*\colon \Tor^{U\lieg}_k(V,
            \CC_\lambda)\neq 0 \text{ for some } k \in \ZZ_{\geq0}\},
    \end{equation}
    where $\CC_\lambda$ denotes a one-dimensional $\lieg$-module, on which $\lieg$ acts by
    multiplication on $\lambda$.
\end{definition}
In further researches there had been shown, that the spectrum has several nice properties, such as
projection property and different variations of the spectral mapping theorem. The details can be
found in the monograph \cite{beltita} and in series of papers by A. Dosi (see for example
\cite{Dosi1}\cite{Dosi2}\cite{Dosi3}). In 2010, A.Dosi proved, that for a special class of Banach
modules over nilpotent Lie algebra, there exists some sort of noncommutative holomorphic functional
calculus \cite{Dosi4}.

In this paper, we introduce the notion of the Taylor joint spectrum for modules over an
arbitary finite-dimensional Lie algebra $\lieg$ and study it for, mostly, finite-dimensional
modules. Our main results are Theorems \ref{t:ssspectrumissubmodules}, \ref{t:nilpspectrum} and
\ref{t:borelspectrum} in which we describe the spectrum for semisimple, nilpotent and Borel
subalgebras respectively.   


\section{Preliminaries}%
\label{sec:preliminaries}
% Functors * and op
% Tensor product of representations
% Chevalley-Eilenberg resolution
% Tor(A, B) = Tor(C, AxB)
% Poincare duality
% 
Let $\lieg$ be an arbitary finite dimensional Lie algebra. We will denote by $\lmod{\lieg}$ and
$\rmod\lieg$ the categories of left and right $\lieg$-modules respectively. We define functors 
$\square^*\colon \lmod{\lieg}^{op} \to \rmod{\lieg}$ and $\square^\circ\colon \lmod{\lieg} \to
\rmod{\lieg}$ as follows. The first $\square^*$, called duality functor, sends $\lieg$-module V to
it's dual vector space, on which the right action of $\lieg$ defined as 
\[
    (f\cdot g) (v) = f(g\cdot v),~\text{ for all } f\in V^*,~v \in V,~g\in\lieg.
\]
The second $\square^\circ$, called antipode functor, sends $V$ to itself as vector space with right
action
\[
    v\cdot g = - g\cdot v,~\text{ for all } v \in V,~g\in\lieg.
\]
These two functors define equivallence of categories $\lmod\lieg$, $\rmod\lieg$, $\lmod\lieg^{op}$
and $\rmod\lieg^{op}$.  


\section{Taylor spectrum of \texorpdfstring{$\lieg$-module}{g-module}}%
\label{sec:spectrumofmodule}
Let $\lieg$ be an arbitary Lie algebra and $V$ be a left $\lieg$-module. 
Recall, that $\hat\lieg$ is the set of isomorphism classes of simple
finite dimensional $\lieg$-modules.
\begin{definition}
    The Taylor spectrum of $V$ is the subset of $\hat\lieg$, defined as
    \[
        \sigma_\lieg(V)=\{ S\in \hat{\lieg} \mid \exists 
            k \colon \Tor^{U\lieg}_k(S^*,V)\neq 0\}.
    \]
\end{definition}
We will simply write $\sigma(V)$ instead of $\sigma_\lieg(V)$ if it is clear what Lie algebra is
considered.

Let us provide several equivalent definitions, in order to use later.
\begin{theorem}
   For an arbitary $S\in\lieg$, the following are equivalent:
   \begin{enumerate}
       \item $S \in \sigma(V)$; 
           
       \item $H_k(\lieg, V_{-S}) \ne 0$ for some $k$;

       \item $H^k(\lieg, \bigwedge^n\lieg\otimes_\CC V_{-S}) \ne 0$ for some $k$;

       \item $S^* \otimes_\CC \bigwedge^n\lieg \in \sigma(V^*)$
   \end{enumerate}
\end{theorem}
\begin{proof}
    \begin{itemize}
        \item[$1 \Rightarrow 2$]
            This follows immediatly from Lemma \ref{t:torhomology}.

        \item[$2 \Rightarrow 3$]
            Poincare duality (Theorem \ref{t:poincare}).

        \item[$3 \Rightarrow 4$] \todo{finish proof}

    \end{itemize}
\end{proof}


\section{Case of semisimple Lie algebras}%
\label{sec:semisimple}
In this section $\lieg$ denotes semisimple Lie algebra of dimension $n$. Note, that $\bigwedge^n
\lieg$ is isomorphic to trivial $\lieg$-module, therefore, by Lemma \ref{l:definitions}, the Taylor
spectrum of $\lieg$-module $V$ can be described as the set
\[
    \sigma_\lieg(V) = \{ S \in \hat\lieg \colon H^k(\lieg, V_{-S})\neq 0 \text{ for some } k\}.
\]
Let us recall the Theorem 7.8.9 from \todo{ref weibel}, that we will need to describe the spectrum
of $\lieg$-modules.
\begin{theorem} \label{t:cohomologyofsemisimple}
If $S$ is a simple $\lieg$-module, $S \neq \cmpl$, then 
\[
    H^k(\lieg, S) = 0 \text{ for all } k.
\]
\end{theorem}
If the module is trivial, then, obviously, $H^k(\lieg, \CC) \cong \CC$. Now we are ready to proove
the main result of the section.
\begin{theorem} \label{t:ssspectrumissubmodules}
    For any finite-dimensional $\lieg$ module $V$, its Taylor spectrum coincides with the set of
    simple components of $V$. Moreover the dimension $H^k(\lieg, V_{-S})$ is the number of copies
    of $S$ in the decomposition of $V$.
\end{theorem}
\begin{proof}
    As the cohomology commute with finite sums it suffices to show, that the assertion of Theorem holds for
    simple $V$. By Theorem \ref{t:cohomologyofsemisimple}, we only need to show that $\CC$ occurs
    in decomposition of $V_{-S}$ if and only if $V\cong S$. But $V_{-S}$ is $\Hom_\CC(S, V)$ and
    $(V_{-S})^\lieg$ is $\Hom_\lieg(S, V)$, so it is one-dimensional if $S\cong V$ and zero
    otherwise.
\end{proof}
In fact, the statement of the above theorem holds even for Banach $\lieg$-modules. In order to
prove this, we need the following fact, which can be found in the book \todo{red beltita}.

\begin{theorem}
    Any Banach module $V$ over semisimple Lie algebra $\lieg$ is the union of its
    finite-dimensional submodules. 
\end{theorem}
\begin{proof}
    Corollary 5 in \S30 of \todo{ref beltita}.
\end{proof}
In other words, $V$ is the colimit of the filtered diagram $\mathfrak{V}$ of its finite-dimensional submodules
with inclusions as morphisms. Recall, that homology and the tensor product commute with filtered
colimits. Thus, we can strengthen the Theorem \ref{t:ssspectrumissubmodules}.
\begin{corollary}
    For any Banach $\lieg$ module $V$, its Taylor spectrum coincides with the set of
    simple submodules of $V$. 
\end{corollary}
\begin{proof}
    As all the maps in $\mathfrak{V}$ are inclusions, so do maps in $H_*(\lieg, \mathfrak{V}_{-S})$,
    hence $H_*(\lieg, V_{-S})$ is nonzero if and only if $S \in \sigma(W)$ for some $W\in
    \mathfrak{V}$.
\end{proof}


\section{Spectrum of one-dimensional extensions}%
\label{sec:1extensions}
In this section we provide a tool for computing spectrum of some modules by induction.

\subsection{Extensions of Lie algebras}
Let $\lieh$ be an arbitary Lie algebra. By one-dimensional extension of $\lieh$ we mean the exact
sequence of Lie algebras
\[
    0 \to \CC_\lambda\rightarrow \lieg \xrightarrow\pi \lieh \to 0,
\]
where $\CC_\lambda$ is one-dimensional $\lieh$-module, such the commutator in $\lieg$ is given by
$[g,c] = \lambda(\pi(g))c$ for all $g \in \lieg$ and $c \in \CC_\lambda$.  The fundamental result of
Lie algebra cohomology theory states that isomorphism classes of such extensions are in one-to-one
correspondence with the set $H^2(\lieh,\CC_\lambda) = \Ext^2_{U\lieh}(\cmpl, \cmpl_\lambda)$. Let
us recall how to construct the bijection. Let $\xi \in Hom_\CC(\bigwedge^2\lieh, \CC_\lambda)$ be a
cocycle. We define Lie bracket on the vector space $\lieg = \CC_\lambda \oplus \lieh$ by 
\[
    [(h_1, v_1), (h_2, v_2)]_\xi= ([h_1, h_2], \lambda(h_1) v_2 - \lambda(h_2) v_1 +
    \xi(h_1\wedge h_2)), 
\]
for all $h_i \in \lieh,~c_i\in \CC_\lambda$. Jacobi identity is obtained from definition of a
cocycle and it can be shown, that cohomologous cocycles induce isomorphic extensions (comp.
\cite{weibel}).

From now on and untill the end of the paragraph, we write $\lieg$ for a one-dimensional extension
of $\lieh$, represented by cocycle $\xi \in Hom_\CC(\bigwedge^2\lieh, \CC_\lambda)$. We also use
the notation $\mathfrak{c}$ for an ideal $\CC_\lambda\subset \lieg$. If $V$ is an $\lieh$ module,
we denote by $V^\pi$ the same vectore space, considered as $\lieg$-module with action determined by
action of $\lieh$ and homomorphism $\pi$. Note, that if $S \in \hat\lieh$ -- ireducible
$\lieh$-module, than $S^\pi$ is also irreducible as $\lieg$-module, thus we will identify
$\hat\lieh$ with subset of $\hat\lieg$. Assume that we have a $\lieh$-module $V$, and we want to
compute it's spectrum $\sigma_\lieg (V^\pi)$, here $V^\pi$ is $V$ considered as $\lieg$-module via
the homomorphism $\pi$. The following Theorem gives us some approximations for the desired result.
\begin{theorem} \label{t:1ext}
    Let $\lieh$, $\lieg$ and $V$ be as above. Then:
    \begin{enumerate}
        \item $\sigma_\lieg(V^\pi) \subset \hat\lieh \subset \hat\lieg$;

        \item For any $S \in \hat\lieh$, $\xi$ induces the maps $\xi_k^* \colon H_k(\lieh, V_{-S})
            \to H_{k - 2}(\lieh,  V_{-S}\otimes_\CC \CC_\lambda)$ for $2 \leq k \leq n$. Moreover, $S \in
            \sigma_\lieg(V^\pi)$ if and only if $\xi^*_k$ is not an isomoprhism for some $k$;

        \item If $S\in \sigma_\lieg(V^\pi)$, then either $S \in \sigma_\lieh(V)$ or
            $S_{-\lambda}\in \sigma_\lieh(V)$.
    \end{enumerate}
\end{theorem}
\begin{proof}
    Let $S \in \hat\lieg \setminus \hat\lieh$. It is easy to verify that $\mathfrak{c}S=\{c\cdot s
    \colon s \in S\, c \in \mathfrak{c}\}$ is a submodule of $S$. Since $S$ is ireducible it is
    either $0$ or $S$. But if it is $0$, then $S$ is actually an $\lieh$ module and it contradicts
    our assumption, so $\mathfrak{c}S = S$ and $S_\mathfrak{c} = 0$. The same argument is used
    to show that the set $S^\mathfrak{c}$ equals to $0$. The Hochschild-Serre spectral sequence
    $E^2_{p, q} = H_p(\lieh, H_q(\mathfrak{c}, V_{-S}))$ convergers to $H_k(\lieg, V_{-S})$, so it
    suffices to prove that $H_q(\mathfrak{c}, V_{- S})= 0$. The only possible nonzero homology
    groups are $H_0(\mathfrak{c}, V_{-S}) = (V_{-S})_\mathfrak{c} \cong (S^*)_\mathfrak{c}.
    \otimes_\CC V = 0$ and $H_1(\mathfrak{c}, V_{-S}) = (V_{-S})^\mathfrak{c} \cong
    (S^*)^\mathfrak{c} \otimes_\CC V = 0$. Therefore $S \not\in \sigma_\lieg(V^\pi)$ by Lemma
    \ref{l:definitions}, which gives us the first assertion.

    Now let $S \in \hat\lieh$. Then, as $\lieh$ modules, the homology groups $H_0(\mathfrak{c},
    V_{-S})$ and $H_1(\mathfrak{c}, V_{-S})$ are isomorphic to $V_{-S}$ and $V\otimes_\CC
    \CC_\lambda$ respectively. The Hochschild-Serre spectral sequence stabilizes on the third page
    and the nonzero differntials in $E^2$ are $d_k\colon H_k(\lieh, V_{-S}) \to H_{k - 2}(\lieh,
    V_{-S}\otimes_\CC \CC_\lambda)$. These differentials can be described as the Yoneda product on
    the class $[\xi] \in \Ext_2(\CC, \CC_\lambda)$ and the spectral sequence collapses if and only
    if all the differentials are isomorphisms of vector spaces.

    The last assertion follows from the fact that if $d_k$ is not an isomoprhism, then either
    $H_{k - 2}(\lieh,  V_{-S}\otimes_\CC \CC_\lambda)$ or $H_k(\lieh, V_{-S})$ is nonzero.
\end{proof}
In the case of central extensions we can strengthen the theorem.
\begin{corollary}
    If $\lieg$ is one-dimensional central extension of $\lieh$ and $V$ is $\lieh$-module, then
    the Taylor spectrum of $V$, regarded as $\lieg$-module, is equal to the spectrum of $V$,
    regarded as $\lieh$-module:
    \[
        \sigma_\lieg(V^\pi) = \sigma_\lieh(V).
    \]

\end{corollary}
\begin{proof}
    By Theorem \ref{t:1ext} we already know, that $\sigma_\lieg(V^\pi) \subset \sigma_\lieh(V)$. On
    the other hand, for any $S\in \sigma_\lieh(V)$ denote by $k_0$ the index of the first non-zero
    homology group $H_*(\lieh, V_{-S})$. Then, the map $\xi^*_{k_0}$, defined in the Theorem, has
    $H_{k_0 - 2}(\lieh, V_{-S}) = 0$ in its' image, and hence is not an isomorphism. Applying the
    second statement of the Theorem \ref{t:1ext}, we deduce the assertion.
\end{proof}
This statement will help us a lot in studying the spectrum of nilpotent Lie algebras, because they are
always can be presented as a sequence of central extensions of an abelian Lie algebra.


\section{Case of solvable Lie algebras}%
\label{sec:solvable}
\subsection{The spectrum of the trivial module}
In this section $\lieg$ will denote an arbitary solvable Lie algebra of dimension $n$. Due to Lie's
theorem, every simple $\lieg$-module is one-dimensional, so we will identify $\hat\lieg$ with the
space of characters $(\lieg/[\lieg,\lieg])^*$. In this case, computing Taylor spectrum is a hard
job even for the trivial module $\CC$. For a finite-dimensional $\lieg$-module $V$ by the set of
weights $\omega(V) \subset \hat\lieg$ we mean the set of diagonal matrix entries in triangular
basis for $V$. It is independant on the choice of upper triangular basis and can be also described
as the set of one-dimensional subfactors of $V$. Consider the adjoint representation $\ad\lieg \in
\lmod\lieg$. The set of weights $\omega(\ad\lieg)$ is called Jordan-H{\"o}lder values of $\lieg$.
We denote by $2\rho$ the sum of all Jordan-H{\"o}lder values with multiplicites. It is exactly the
weight of the module $\bigwedge^n\lieg$, defined in section \ref{sec:preliminaries}.  The following
restriction on possible elements in the spectrum of the trivial module holds.
\begin{theorem} \label{t:jordanholder}
    For any solvable Lie algebra $\lieg$ of dimension $n$, if $\lambda \in \hat\lieg$ is in the
    spectrum $\sigma_\lieg(\CC)$, then it is the sum of at most $n$ Jordan-H{\"o}lder values of
    $\lieg$. Moreover, if $\lambda \in \sigma_\lieg(\CC)$, then $2\rho - \lambda$ is also in
    $\sigma_\lieg(\CC)$.
\end{theorem}
\begin{proof}
    The proof is by induction on $n$. Assertion of the Theorem, obviously, holds for
    one-dimensional Lie Algebra, so we have $n=1$ as the base of induction. Assume, that it holds
    for all solvable Lie algebras of dimension $n-1$. Choose any one-dimensional ideal
    $\mathfrak{c}$ in $\lieg$ and denote the corresponding character by $\mu$. By the third
    statement of the Theorem \ref{t:1ext} we know, that the spectrum $\sigma_\lieg(\CC)$ lies in
    the set $\{0, \mu\} + \sigma_{\lieg/\mathfrak{e}}(\CC)$. By induction, any $\nu \in
    \sigma_{\lieg/\mathfrak{e}}(\CC)$ is the sum of at most $n-1$ Jordan-H{\"o}lder values of
    $\lieg/\mathfrak{e}$, which are also Jordan-H{\"o}lder values of $\lieg$. This is the desired
    conclusion.

    For the second assertion we use Lemma \ref{l:definitions}. The trivial module is isomorphic to
    its dual, so if $\CC_\lambda$ is in the spectrum of $\CC$, then $\CC_\lambda^* \otimes
    \bigwedge^n\lieg \cong_\CC \CC_{2\rho - \lambda}$ is also in the spectrum.
\end{proof}
Obviously, $0$ is always in the spectrum and, hence, so is $2\rho$. So, generally, there are often
more than one element in the spectrum of $\CC$. Nevertheless, for some good classes of solvable Lie
algebras the situation is more clear, than in general case. One of such classes is nilpotent Lie
algebras.

\subsection{The spectrum of finite-dimensional modules over nilpotent Lie algebras}
The following well-known fact in representation theory for nilpotent Lie algebras plays major role
in this paragraph.
\begin{lemma} \label{l:nilpdecomp}
    For $V$ -- finite-dimensional indecomposable module over nilpotent Lie algebra $\lieg$,
    $\omega(V)$ consists of one element.
\end{lemma}
\begin{proof}
    Proposition 9 in chapter VII of \todo{ref bourbaki}
\end{proof}
We call modules with only one weight monoweighted. The last Lemma show us, that any
finite-dimensional module over nilpotent Lie algebra can be decomposed in sum of monoweighted
submodules.  Now we are ready to formulate the main result.
\begin{theorem}
    Let $\lieg$ be a nilpotent Lie algebra and $V$ -- finite-dimensional $\lieg$-module. Then the
    spectrum $\sigma(V)$ coincides with the set of weights $\omega(V)$.
\end{theorem}
\begin{proof}
    As in the proof of \ref{t:ssspectrumissubmodules} we may assume $V$ is monoweighted. The
    spectrum of module $V_{\lambda}$ is $\sigma(V) + \lambda$, so we may additionally assume, that
    $\omega(V) = \{0\}$. Observe, that by Engel's theorem, all Jordan-H{\"o}lder values of $\lieg$
    are zero, so by Theorem \ref{t:jordanholder}, we conclude that the assertion holds for trivial
    $V$. We now proceed by induction on $m = \dim V$. Choose some one-dimensional submodule in $V$.
    We have a short exact sequence
    \[
        0\to \CC \to V \to V/\CC \to 0.
    \]
    It follows easily from long exact sequence of cohomologies, that $\sigma(V) \subset
    \sigma(\CC) \cup \sigma(V/\CC)$ and the latest is equal $\{0\}$ by induction. On the other hand
    $0$ is in the spectrum of $V$, because $H_0(\lieg, V) = V_\lieg \neq 0$. This finishes the
    proof.
\end{proof}

\subsection{Case of borel subalgebras of semisimple Lie algebras}
%Пусть $\lies$ -- полупростая алгебра Ли и $\lieg$ -- некоторая её борелевская подалгебра. Известно, что $\lieg = \lieh \ltimes \lien$, где $\lieh$ -- картановская подалгебра $\lies$, а $\lien = [\lieg, \lieg]$. Введем обозначение $\Delta, \Delta^+ \subset \lieh^* = \lieg^*$ для системы корней относительно $\lieh$ и множества положительных корней относительно $\lieg$ соответственно. В терминах спектра Тейлора $\Delta = \sigma_h(\lies)$ и $\Delta^+ = \sigma_h(\lien)$ (так как $\lieh$ -- абелева, то это просто множество совместных собственных значений). За $W$ обозначим группу Вейля относительно $\Delta^+$. Для каждого элемента $w\in W$ определена его длина $0 \leq l(w) \leq \dim \lien$

%Нашей задачей является описание спектра Тейлора тривиального модуля алгебры Ли $\lieg$. Заметим, что эта задача эквивалентна следующей: для каких одномерных $\lieg$-модулей $\cmpl_\lambda$ его гомологии $H^{\mathrm{Lie}}_{\bullet}(\lieg, \cmpl_\lambda)$ не зануляются? Далее мы приведем полное решение этой задачи, более того мы получим явный вид соответсвующих гомологий. 
Let $\lies$ be a semisimple Lie algeba and $\lieg$ its Borel subalgebra. It is known\todo{ref
humphreys}, that $\lieg$ is isomorphic to semidirect product $\lieh \ltimes \lien$, where $\lieh$ is
a Cartan subalgebra of $\lies$ and $\lien = [\lieg, \lieg]$. By $\Delta \subset \lieh^* =
\hat\lieg$ we will denote the root system of $\lies$ relative to $\lieh$ and $\lieg$. We write
$\Delta^+ \subset \Delta$ for denote the subset of positive roots. In fact, elements of $\Delta^+$
are nonzero Jordan-H\"older values of $\lieg$. Let $W(\Delta)$ denote the Weil group. For any
element $w\in W$ its length is denoted by $l(w)$. 

The aim of the paragraph is to describe the spectrum of irreducible $\lies$ modules, regarded
as $\lieg$ modules. The plan is to compute relveant cohomologies of $\lien$ and then take advantage
of the Hochshild-Serre spectral sequence. Fortunately, these two steps were alredy done by
different authors and we need only to glue them together. The first one is Kostant Theorem.
\begin{theorem}[Kostant]
   Let $V$ be irreducible representation of semisiple Lie algebra $\lies$ with highest weight
   $\lambda$, relative to borel subalgebra $\lieg = \lieh \ltimes \lien$. Then, as $\lieh$-module,
   $H^k(\lien, V)$ is the sum of one-dimensional modules of weights
   \[
       w(\lambda + \rho) - \rho,~\in W(\Delta), ~l(w) = k,
   \]
   where $\rho$ is half-sum of positive roots (or, equivalently, Jordan-H\"older values).
\end{theorem}
\begin{proof}
    Theorem 6.12 in \todo{cite knapp} 
\end{proof}
We say, that an abelian Lie algebra $\lieh$ acts torally on $\lieh$-module $W$ if $W$ is direct sum
of one-dimensional submodules. When $\lieh$ is a Cartan subalgebra of semisiple Lie algebra
$\lies$, $\lieh$ acts torally on any finite dimensional $\lies$-module. If $\lieg = \lieh \ltimes
\lien$ -- the Borel subalgebra of $\lies$, then $\lieh$ also acts torally on $\lien$. The following
Theorem guarantees convergence of the Hochsild-Serre spectral sequence in the special case of
semidirect products.
\begin{theorem}
    Let $\lieg=\lieh\ltimes\lien$ be a semidirect product where $\lieh$ is abelian and acts torally
    both on $\lieg$ and a $\lieg$-module $V$. Then with respect to the weighting induced by the
    action of $\lieh$ we have 
    \begin{equation} \label{t:coll:1}
        H^k(\lieg, V) = \bigoplus_{p + q = k} \bigwedge^p\lieh^* \otimes_\CC H^q(\lien, V)^\lieh
    \end{equation} 
\end{theorem}
\begin{proof}
    Theorem 4 in \todo{ref coll} 
\end{proof}

Let $\mu \in \lieh^*$ a weight. For any $\lieg$-module $V$, one may observe, that $H^k(\lien,
V_{-\mu}) \cong H^k(\lien, V)_{-\mu}$ for all $k$. Hence, if $V$ is irreducible $\lies$-module of
highest weight $\lambda$, then $H^q(\lien, V)$ from \ref{t:coll:1} is nonzero for some $q$ if and only if $\mu =
w(\lambda + \rho) - \rho$. Combining it with Lemma $\ref{l:definitions}$ we obtain the main result.
\begin{theorem}
   For an irreducible module $V$ of highest weight $\lambda$ over the semisimple Lie algebra
   $\lies$, the Taylor spectrum of $V$, regarded as the module over Borel subalgebra $\lieg \subset
   \lies$ has the form 
   \[
       \sigma_\lieg(V) = \{ \rho + w(\lambda + \rho)\colon w \in W(\Delta)\}
   \]
\end{theorem}


%\section{Case of nilpotent Lie algebras}
%\label{sec:nilpotent}
%\input{nilpotent.tex}


%\section{Case of borel subalgebra of semisimple Lie algebra}
%\label{sec:borel}
%\input{borel.tex}
\printbibliography

\end{document}
