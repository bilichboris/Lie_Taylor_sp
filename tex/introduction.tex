In 1970, J. Taylor\cite{Taylor1970} introduced the notion of the joint spectrum for a finite tuple
of commuting operators $T = (T_1,...,T_n)$, acting on a Banach space $V$. It was defined as a
subset of $\CC^n$ (where $n$ is number of operators), which consists of such elements
$(\lambda_1,...,\lambda_n)$, that the Koszul complex for algebra $\mathcal{B}(V)$ of bounded
operators and elements $T_1 - \lambda_1,...,T_n - \lambda_n$ is not exact. For one operator, the
Taylor spectrum coincides with the classical one. In the same year, Taylor estabilished the
existence of the holomorphic functional calculus in the neighbourhood of the spectrum
\cite{Taylor1970b}. Two years later, he proposed a framework for noncommutative functional
calculus\cite{Taylor1972}, but the notion of spectrum for non-commuting tuples of operators was not
yet developed. 

The step in this direction was made by A.S. Fainshtein in his work \cite{Fainshtein}, where he
generalized the Taylor spectrum to tuples of operators, generating nilpotent Lie algebra $\lieg$.
This definiton was soon improved by E. Boasso and A. Larotonda \cite{boasso} to fit also for solvable Lie
algebras. Let us recall the definiton, given by them.
\begin{definition}
    Let $\lieg$ be a solvable Lie algebra, and $V$ be a right $\lieg$-module. The Taylor spectrum
    of $V$ is the subset of the space of characters $(\lieg/[\lieg,\lieg])^*$ of the form
    \begin{equation} \label{e:boassolarotonda}
        \sigma^{BL}_\lieg(V) = \{ \lambda \in (\lieg/[\lieg,\lieg])^*\colon \Tor^{U\lieg}_k(V,
            \CC_\lambda)\neq 0 \text{ for some } k \in \ZZ_{\geq0}\},
    \end{equation}
    where $\CC_\lambda$ denotes a one-dimensional $\lieg$-module, on which $\lieg$ acts by
    multiplication on $\lambda$.
\end{definition}
In further researches there had been shown, that the spectrum has several nice properties, such as
projection property and different variations of the spectral mapping theorem. The details can be
found in the monograph \cite{beltita} and in series of papers by A. Dosi (see for example
\cite{Dosi1}\cite{Dosi2}\cite{Dosi3}). In 2010, A.Dosi proved, that for a special class of Banach
modules over nilpotent Lie algebra, there exists some sort of noncommutative holomorphic functional
calculus \cite{Dosi4}.

In this paper, we introduce the notion of the Taylor joint spectrum for modules over an
arbitary finite-dimensional Lie algebra $\lieg$ and study it for, mostly, finite-dimensional
modules. Our main results are Theorems \ref{t:ssspectrumissubmodules}, \ref{t:nilpspectrum} and
\ref{t:borelspectrum} in which we describe the spectrum for semisimple, nilpotent and Borel
subalgebras respectively.   
