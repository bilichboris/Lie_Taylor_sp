% Functors * and op
% Tensor product of representations
% Chevalley-Eilenberg resolution
% Tor(A, B) = Tor(C, AxB)
% Poincar\'e duality
% 

\subsection{Notation} 
In the article all algebras, including Lie algebras, are complex.  For Lie algebra $\lieg$ we will
use the notation $U\lieg$ to denote its enveloping algebra. We will denote by $\lmod{\lieg}$ and
$\rmod\lieg$ the categories of left and right $\lieg$-modules respectively.  We write $\hat\lieg$
for the set of isomorphism classes of simple finite-dimensional $\lieg$-modules and $\cmpl$
for trivial bimodule.  For $\lieg$-module $V$ one can define vector spaces 
\begin{equation}
    V^\lieg = \{v\in V\colon g\cdot v = 0 ~ \forall g\in \lieg \}, 
\end{equation}
called invariants and
\begin{equation}
    V_\lieg = V/gV,
\end{equation}
called coinvariants. It is known \cite{weibel}, that $\square^\lieg$ and $\square_\lieg$ are actually
functors from $\lmod\lieg$ (or $\rmod\lieg$) to the category of vector spaces over $\CC$,
isomorphic to $\Hom_\lieg(\CC, V)$ and $\CC \otimes_{U\lieg} V$ respectively. 

\subsection{Functors between categories of modules}
For the rest of this section we will denote by $\lieg$ an arbitary finite-dimensional Lie algebra.
We define two functors $\square^*\colon \lmod{\lieg}^{op} \to \rmod{\lieg}$ and
$\square^\circ\colon \lmod{\lieg} \to \rmod{\lieg}$ as follows. The first $\square^*$, called
duality functor, sends a $\lieg$-module V to it's dual vector space, on which the right action of
$\lieg$ is defined as 
\[
    (f\cdot g) (v) = f(g\cdot v),~\text{ for all } f\in V^*,~v \in V,~g\in\lieg.
\]
The second $\square^\circ$, called antipode functor, sends $V$ to itself as a vector space with right
action
\[
    v\cdot g = - g\cdot v,~\text{ for all } v \in V,~g\in\lieg.
\]
These two functors define equivallence of categories $\lmod\lieg$, $\rmod\lieg$, $\lmod\lieg^{op}$
and $\rmod\lieg^{op}$. We will also denote by $\square^*$ and $\square^\circ$ functors from
category of right $\lieg$-modules to left $\lieg$-modules, defined in the same way. It is easy to see,
that $(\square^*)^*$ and $(\square^\circ)^\circ$ are naturally isomorphic to the identity functor.

Another very important functor is $\square \otimes_\cmpl \square \colon \rmod\lieg \times
\lmod\lieg \to \lmod\lieg$. If $V \in \rmod\lieg$ and $W \in \lmod\lieg$, then $V \otimes_\cmpl W$
is the tensor product of $V$ and $W$ as vector space with action of $\lieg$, fully determined by
the formula
\[
    g\cdot v\otimes w = v\otimes (g\cdot w) - (v\cdot g) \otimes w, \text{ for all } w\in W,~v\in
    V,~ g\in \lieg.
\]
For $V, W \in \lmod\lieg$ (resp. $\rmod\lieg$), we will denote by $V\otimes W$ left $\lieg$-module
$V^\circ \otimes_\CC W$ (resp. $V \otimes_\CC W^\circ$).

For $V\in \lmod\lieg$ and $S \in \hat\lieg$, we will write $V_{S}$ for the $\lieg$-module
$S\otimes_\CC V$.  If S is one-dimensional, it is fully determined by the character $\lambda \in
(\lieg/[\lieg, \lieg])$ and in this case we will simply write $V_\lambda$ for $V_S$. For example,
$\CC_\lambda$ stands for one-dimensional module with action, given by $g\cdot s = \lambda(g)s$ for
all $s\in \CC_\lambda$ and $g \in \lieg$. We will also use the notation $V_{-S}$ for the module
$S^* \otimes_\CC V$, which is motivated by the fact, that if $S$ is again one-dimensional with
character $\lambda$, then $V_{-S}$ is isomorphic to $V_{-\lambda}$. 

\subsection{Homology and cohomology of Lie algebras}
In this paragraph we recall the definitions of Lie algebra cohomology, which can be found in any
related textbook (for example \cite{weibel}).


\begin{definition}
    For $V\in \lmod\lieg$ and for all $k\in \ZZ_{\geq0}$ the homology functors are defined as 
    \begin{equation}
        H_k(\lieg, V) = \Tor_k^{U\lieg}(\cmpl, V),
    \end{equation}
    and, dually, the cohomology as
    \begin{equation}
        H^k(\lieg, V) = \Ext^k_{U\lieg}(\CC, V).
    \end{equation}
\end{definition}

The homology can be computed using Chevalley-Eilenberg \cite{guichardet} free
resolution of the trivial $\lieg$ module $\cmpl$. It has $F_k=U\lieg\otimes_\cmpl \bigwedge^k
\lieg$ in degree $k$ with the differential given by
\begin{equation}
    d(u\otimes g_1 \wedge \cdots \wedge g_p) 
 = \sum_{i = 1}^p (-1)^{i+1} u g_i \otimes g_1 \wedge \cdots \wedge \hat{g}_i\wedge \cdots \wedge g_p  
+$$ $$+\sum_{i < j} (-1)^{i+j} u\otimes [g_i, g_j] \wedge \cdots \wedge \hat{g}_i\cdots \wedge
\hat{g}_j\cdots \wedge g_p,~\text{where } u\in U\lieg,~g_i \in \lieg.
\end{equation}
The following fact about homology will be used in the text.
\begin{lemma} \label{t:torhomology}
   Let $V \in \rmod\lieg$ and $W \in \lmod\lieg$. Then  
    \[
        \Tor_k^{U\lieg}(V,W) \cong \Tor_k^{U\lieg}(\cmpl,V\otimes_\cmpl W) = H_k(\lieg, V\otimes_\cmpl W),
    \]
    for all $k\in \ZZ_{\geq 0}$.
\end{lemma}
\begin{proof}
    Since the functors $\CC\otimes_{U\lieg}(\square \otimes_\CC W)$ and $\square \otimes_{U\lieg} W$
    are naturally isomorphic, it suffices to show, that if $P_\bullet \to V$ is a flat resolution
    of $V$, then $P_\bullet \otimes_\CC W$ is a flat resolution of $V\otimes_\CC W$.

    By definition, flatness of $P_k$ means exactness of functor $P_k\otimes_{U\lieg}\square$. Using
    properties of tensor product we obtain an isomorphism of functors $(P_k\otimes_\CC W)
    \otimes_{U\lieg} \square$ and $P_k\otimes_{U\lieg} (W \otimes_\CC \square)$. The last functor
    is the composition of exact functors, so it is also exact, hence $P_k\otimes_\CC W$ is flat.
\end{proof}

A useful variation of Poincar\'e duality holds for finite dimensional Lie algebras. Let $n = \dim
\lieg$. We endow the one-dimensional vector space $\bigwedge^n \lieg$ with structure of left
$\lieg$-module, which extends adjoint action by Leibnitz rule. 

\begin{theorem}[Poincar\'e duality] \label{t:poincare}
    For $0 \leq k \leq n$ and any left module $V$ over Lie algebra $\lieg$ of dimension $n$, there
    are vector space isomorphisms 
    \[
        H^k(\lieg, V^*) \cong H_k(\lieg, V)^*,
    \]
    and
    \[
        H^k(\lieg, V) \cong H_{n - k}(\lieg, (\bigwedge^n \lieg)^*\otimes_\cmpl V),
    \]
    natural in V. Consequently
    \[
        H^k(\lieg, V^*) \cong H^{n - k}(\lieg,  \bigwedge^n \lieg\otimes_\cmpl V)^*. 
    \]
\end{theorem}
\begin{proof}
    Theorem 6.10 in \cite{knapp}.
\end{proof}
