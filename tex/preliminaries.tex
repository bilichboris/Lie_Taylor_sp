% Functors * and op
% Tensor product of representations
% Chevalley-Eilenberg resolution
% Tor(A, B) = Tor(C, AxB)
% Poincare duality
% 

\subsection{Notation} \todo{$\CC_\lambda$}
In the article all algebras, including Lie algebras, are complex. 
For the rest of this section we will denote by $\lieg$ an arbitary Lie algebra. We will
denote by $\lmod{\lieg}$ and $\rmod\lieg$ the categories of left and right $\lieg$-modules respectively. 
For $\lieg$-module $V$ defined the spaces 
\begin{equation}
    V^\lieg = \{v\in V\colon g\cdot v = 0 ~ \forall g\in \lieg \}, 
\end{equation}
called invariants and
\begin{equation}
    V_\lieg = V/gV,
\end{equation}
called coinvariants. It is known \todo{ref}, that $\square^\lieg$ and $\square_\lieg$ are actually functors
from $\lmod\lieg$ (or $\rmod\lieg$) to the category of vector spaces over $\CC$.

\subsection{Functors between categories of modules}
We define two functors $\square^*\colon \\ \lmod{\lieg}^{op} \to \rmod{\lieg}$ and $\square^\circ\colon
\lmod{\lieg} \to \rmod{\lieg}$ as follows. The first $\square^*$, called duality functor, sends 
$\lieg$-module V to it's dual vector space, on which the right action of $\lieg$ is defined as 
\[
    (f\cdot g) (v) = f(g\cdot v),~\text{ for all } f\in V^*,~v \in V,~g\in\lieg.
\]
The second $\square^\circ$, called antipode functor, sends $V$ to itself as a vector space with right
action
\[\label{test}
    v\cdot g = - g\cdot v,~\text{ for all } v \in V,~g\in\lieg.
\]
These two functors define equivallence of categories $\lmod\lieg$, $\rmod\lieg$, $\lmod\lieg^{op}$
and $\rmod\lieg^{op}$. We will also denote by $\square^*$ and $\square^\circ$ functors from
category of right $\lieg$-modules to left $\lieg$-modules, defined the same way. It is easy to
see, that $(\square^*)^*$ and $(\square^\circ)^\circ$ are naturally isomorphic to the identity functor.

Another pair of very important functors are $\square \otimes_\cmpl \square \colon \rmod\lieg \times
\lmod\lieg \to \lmod\lieg$ and $\Hom_\cmpl(\square, \square)\colon \lmod\lieg^{op} \times
\lmod\lieg \to \lmod\lieg$. If $V \in \rmod\lieg$ and $W \in \lmod\lieg$, then $V \otimes_\cmpl W$
is the tensor product of $V$ and $W$ as vector space with action of $\lieg$, fully determined by the
formula
\[
    g\cdot v\otimes w = v\otimes (g\cdot w) - (v\cdot g) \otimes w, \text{ for all } w\in W,~v\in
    V,~ g\in \lieg 
\].
The Hom functor is defined as 
\[
    \Hom_\cmpl(V, W)=V^* \otimes_\cmpl W.
\]


\subsection{Homology and cohomology of Lie algebras}
In this paragraph we recall the definitions of Lie algebra cohomology, which can be found in any
related textbook \todo{ref weibel}.

\begin{definition}
    For $V\in \lmod\lieg$ and for all $i\in \ZZ_{\geq0}$ the homology functors are defined as 
    \begin{equation}
        H_i(\lieg, V) = \Tor_i^{U\lieg}(\cmpl, V),
    \end{equation}
    and, dually, the cohomology as
    \begin{equation}
        H^i(\lieg, V) = \Ext^i_{U\lieg}(\CC, V).
    \end{equation}

\end{definition}
