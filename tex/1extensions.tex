In this section we provide a tool for computing spectrum of some modules by induction.

\subsection{Extensions of Lie algebras}
Let $\lieh$ be an arbitary Lie algebra. By one-dimensional extension of $\lieh$ we mean the exact
sequence of Lie algebras
\[
    0 \to \CC_\lambda\rightarrow \lieg \xrightarrow\pi \lieh \to 0,
\]
where $\CC_\lambda$ is one-dimensional $\lieh$-module, such the commutator in $\lieg$ is given by
$[g,c] = \lambda(\pi(g))c$ for all $g \in \lieg$ and $c \in \CC_\lambda$.  The fundamental result of
Lie algebra cohomology theory states that isomorphism classes of such extensions are in one-to-one
correspondence with the set $H^2(\lieg,\CC_\lambda) = \Ext^2_{U\lieh}(\cmpl, \cmpl_\lambda)$. Let
us recall how to construct the bijection. Let $\xi \in Hom_\CC(\bigwedge^2\lieg, \CC_\lambda)$ be a
cocycle. We define Lie bracket on the vector space $\lieg = \CC_\lambda \oplus \lieh$ by 
\[
    [(h_1, v_1), (h_2, v_2)]_\xi= ([h_1, h_2], \lambda(h_1) v_2 - \lambda(h_2) v_1 +
    \xi(h_1\wedge h_2)), 
\]
for all $h_i \in \lieh,~c_i\in \CC_\lambda$. Jacobi identity is obtained from definition of a
cocycle and it can be shown, that cohomologous cocycles induce isomorphic extensions (comp.
\cite{weibel}).

From now on and untill the end of the paragraph, we write $\lieg$ for a one-dimensional extension
of $\lieh$, represented by cocycle $\xi \in Hom_\CC(\bigwedge^2\lieg, \CC_\lambda)$. We also use
the notation $\mathfrak{c}$ for an ideal $\CC_\lambda\subset \lieg$. Note, that if $S \in
\hat\lieh$ -- ireducible $\lieh$-module, than $S^\pi$ is also irreducible as $\lieg$-module, thus
we will identify $\hat\lieh$ with subset of $\hat\lieg$. Assume that we have a $\lieh$-module $V$,
and we want to compute it's spectrum $\sigma_\lieg (V^\pi)$, here $V^\pi$ is $V$ considered as
$\lieg$-module via the homomorphism $\pi$. The following Theorem gives us some approximations for
the desired result.
\begin{theorem} \label{t:1ext}
    Let $\lieh$, $\lieg$ and $V$ be as above. Then:
    \begin{enumerate}
        \item $\sigma_\lieg(V^\pi) \subset \hat\lieh \subset \hat\lieg$;

        \item For any $S \in \hat\lieh$, $\xi$ induces the maps $\xi_k^* \colon H_k(\lieh, V_{-S})
            \to H_{k - 2}(\lieh,  V_{-S}\otimes_\CC \CC_\lambda)$ for $2 \leq k \leq n$. Moreover, $S \in
            \sigma_\lieg(V^\pi)$ if and only if $\xi^*_k$ is not an isomoprhism for some $k$;

        \item If $S\in \sigma_\lieg(V^\pi)$, then either $S \in \sigma_\lieh(V)$ or
            $S_{-\lambda}\in \sigma_\lieh(V)$.
    \end{enumerate}
\end{theorem}
\begin{proof}
    Let $S \in \hat\lieg \setminus \hat\lieh$. It is easy to verify that $\mathfrak{c}S=\{c\cdot s
    \colon s \in S\, c \in \mathfrak{c}\}$ is a submodule of $S$. Since $S$ is ireducible it is
    either $0$ or $S$. But if it is $0$, then $S$ is actually an $\lieh$ module and it contradicts
    our assumption, so $\mathfrak{c}S = S$ and $S_\mathfrak{c} = 0$. The same argument is used
    to show that the set $S^\mathfrak{c}$ equals to $0$. The Hochsild-Serre spectral sequence
    $E^2_{p, q} = H_p(\lieh, H_q(\mathfrak{c}, V_{-S}))$ convergers to $H_k(\lieg, V_{-S})$, so it
    suffices to prove that $H_q(\mathfrak{c}, V_{- S})= 0$. The only possible nonzero homology
    groups are $H_0(\mathfrak{c}, V_{-S}) = (V_{-S})_\mathfrak{c} \cong (S^*)_\mathfrak{c}.
    \otimes_\CC V = 0$ and $H_1(\mathfrak{c}, V_{-S}) = (V_{-S})^\mathfrak{c} \cong
    (S^*)^\mathfrak{c} \otimes_\CC V = 0$. Therefore $S \not\in \sigma_\lieg(V^\pi)$ by Lemma
    \ref{l:definitions}, which gives us the first assertion.

    Now let $S \in \hat\lieh$. Then, as $\lieh$ modules, the homology groups $H_0(\mathfrak{c},
    V_{-S})$ and $H_1(\mathfrak{c}, V_{-S})$ are isomorphic to $V_{-S}$ and $V\otimes_\CC
    \CC_\lambda$ respectively. The Hochsild-Serre spectral sequence stabilizes on the third page
    and the nonzero differntials in $E^2$ are $d_k\colon H_k(\lieh, V_{-S}) \to H_{k - 2}(\lieh,
    V_{-S}\otimes_\CC \CC_\lambda)$. These differentials can be described as the Yoneda product on
    the class $[\xi] \in \Ext_2(\CC, \CC_\lambda)$ and the spectral sequence collapses if and only
    if all the differentials are isomorphisms of vector spaces.

    The last assertion follows from the fact that if $d_k$ is not an isomoprhism, then either
    $H_{k - 2}(\lieh,  V_{-S}\otimes_\CC \CC_\lambda)$ or $H_k(\lieh, V_{-S})$ is nonzero.
\end{proof}
In case of central extensions we can strengthen the theorem.
\begin{corollary}
    If $\lieg$ is one-dimensional central extension of $\lieh$ and $V$ is $\lieh$-module, then
    the Taylor spectrum of $V$, regarded as $\lieg$-module, is equal to the spectrum of $V$,
    regarded as $\lieh$-module:
    \[
        \sigma_\lieg(V^\pi) = \sigma_\lieh(V).
    \]

\end{corollary}
\begin{proof}
    By Theorem \ref{t:1ext} we already know, that $\sigma_\lieg(V^\pi) \subset \sigma_\lieh(V)$. On
    the other hand, for any $S\in \sigma_\lieh(V)$ denote by $k_0$ the index of the first non-zero
    homology group $H_*(\lieh, V_{-S})$. Then, the map $\xi^*_{k_0}$, defined in the Theorem, has
    $H_{k_0 - 2}(\lieh, V_{-S}) = 0$ in its' image, and hence is not an isomorphism. Applying the
    second statement of the Theorem \ref{t:1ext}, we deduce the assertion.
\end{proof}
This statement will help us a lot in studying the spectrum of nilpotent Lie algebras, because they are
always can be presented as a sequence of central extensions of an abelian Lie algebra.
