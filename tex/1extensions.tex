In this section we provide a tool for computing spectrum of some modules by induction.

\subsection{Extensions of Lie algebras}
Let $\lieh$ be an arbitary Lie algebra. By one-dimensional extension of $\lieh$ we call the exact
sequence of Lie algebras
\[
    0 \to \CC_\lambda\rightarrow \lieg \xrightarrow\pi \lieh \to 0,
\]
where $\CC_\lambda$ is one-dimensional $\lieh$-module, such the commutator in $\lieg$ is given by
$[g,c] = \lambda(\pi(g))c$ for all $g \in \lieg$ and $c \in \CC_\lambda$.  The fundamental result of
Lie algebra cohomology theory states, that isomorphism classes of such extensions are in one-to-one
correspondence with the set $H^2(\lieg,\CC_\lambda) = \Ext^2_{U\lieh}(\cmpl, \cmpl_\lambda)$. Let
us recall how to construct the bijection. Let $\xi \in Hom_\CC(\bigwedge^2\lieg, \CC_\lambda)$ be a
cocycle. We define Lie bracket on the vector space $\lieg = \CC_\lambda \oplus \lieh$ by 
\[
    [(h_1, v_1), (h_2, v_2)]_\xi= ([h_1, h_2], \lambda(h_1) v_2 - \lambda(h_2) v_1 +
    \xi(h_1\wedge h_2)), 
\]
for all $h_i \in \lieh,~c_i\in \CC_\lambda$. Jacobi identity is obtained from definition of cocycle
and it can be shown, that cohomologous cocycles induce isomorphic extensions (comp. \todo{ref
weibel}).

From now and till the end of the paragraph, we write $\lieg$ for one-dimensional extension of
$\lieh$, represented by cocycle $\xi \in Hom_\CC(\bigwedge^2\lieg, \CC_\lambda)$. Note, that if $S \in
\hat\lieh$ -- ireducible $\lieh$-module, than $S^\pi$ is also irreducible as $\lieg$-module, thus
we will identify $\hat\lieh$ with subset of $\hat\lieg$. Assume that we
have a $\lieh$-module $V$, and we want to compute it's spectrum $\sigma_\lieg (V^\pi)$, here
$V^\pi$ is $V$ considered as $\lieg$-module via the homomorphism $\pi$. The following Theorem gives
us some approximations for the desired result.
\begin{theorem}
    Let $\lieh$, $\lieg$ and $V$ be as above. Then,
    \begin{enumerate}
        \item $\sigma_\lieg(V^\pi) \subset \hat\lieh \subset \hat\lieg$,

        \item For any $S \in \hat\lieh$, $\xi$ induces the maps $\xi_k^* \colon H_k(\lieh, V_{-S})
            \to H_{k - 2}(\lieh,  V_{-S}\otimes_\CC \CC_\lambda)$. Moreover, $S \in
            \sigma_\lieg(V^\pi)$ if and only if $\xi^*_k$ is not an isomoprhism for some $k$.
            Hence, if $S\in \sigma_\lieg(V^\pi)$, then either $S \in \sigma_\lieh(V)$ or
            $S_{-\lambda}\in \sigma_\lieh(V)$.
    \end{enumerate}
\end{theorem}
\begin{proof}
    \todo{}
\end{proof}
