In this section $\lieg$ denotes semisimple Lie algebra of dimension $n$. Note, that $\bigwedge^n
\lieg$ is isomorphic to trivial $\lieg$-module, therefore, by Lemma \ref{l:definitions}, the Taylor
spectrum of $\lieg$-module $V$ can be described as the set
\[
    \sigma_\lieg(V) = \{ S \in \hat\lieg \colon H^k(\lieg, V_{-S})\neq 0 \text{ for some } k\}.
\]
Let us recall the Theorem 7.8.9 from \cite{weibel}, that we will use to describe the spectrum
of $\lieg$-modules.
\begin{theorem} \label{t:cohomologyofsemisimple}
If $S$ is a simple module over the semisimple Lie algebra $\lieg$, $S \neq \cmpl$, then 
\[
    H^k(\lieg, S) = 0 \text{ for all } k.
\]
\end{theorem}
If the module is trivial, then, obviously, $H^0(\lieg, \CC) \cong \CC$. Now we are ready to proove
the main result of the section.
\begin{theorem} \label{t:ssspectrumissubmodules}
    For any finite-dimensional $\lieg$ module $V$, its Taylor spectrum coincides with the set of
    simple components of $V$. Moreover the dimension of $H^0(\lieg, V_{-S})$ is the number of copies
    of $S$ in the decomposition of $V$.
\end{theorem}
\begin{proof}
    As the cohomology commute with finite sums it suffices to show, that the assertion of Theorem holds for
    simple $V$. By Theorem \ref{t:cohomologyofsemisimple}, we only need to show that $\CC$ occurs
    in decomposition of $V_{-S}$ if and only if $V\cong S$. But $V_{-S}$ is $\Hom_\CC(S, V)$ and
    $(V_{-S})^\lieg$ is $\Hom_\lieg(S, V)$, so it is one-dimensional if $S\cong V$ and zero
    otherwise.
\end{proof}
In fact, the statement of the theorem above holds even for Banach $\lieg$-modules. In order to
prove this, we need the following fact, which can be found in the book \cite{beltita}.

\begin{theorem}
    Any Banach module $V$ over semisimple Lie algebra $\lieg$ is the union of its
    finite-dimensional submodules. 
\end{theorem}
\begin{proof}
    Corollary 5 in \S30 of \cite{beltita}.
\end{proof}
In other words, $V$ is the colimit of the filtered diagram $\mathfrak{V}$ of its finite-dimensional submodules
with inclusions as morphisms. Recall, that homology and the tensor product commute with filtered
colimits. Thus, we can strengthen the Theorem \ref{t:ssspectrumissubmodules}.
\begin{corollary}
    For any Banach $\lieg$ module $V$, its Taylor spectrum coincides with the set of
    simple submodules of $V$. 
\end{corollary}
\begin{proof}
    As all the maps in $\mathfrak{V}$ are inclusions, so are maps in $H_*(\lieg, \mathfrak{V}_{-S})$,
    hence $H_*(\lieg, V_{-S})$ is nonzero if and only if $S \in \sigma(W)$ for some $W\in
    \mathfrak{V}$.
\end{proof}
