Let $\lieg$ be an arbitary Lie algebra of dimension $n$ and $V$ be a left $\lieg$-module.  Recall, that $\hat\lieg$
is the set of isomorphism classes of simple finite dimensional $\lieg$-modules.
\begin{definition}
    The Taylor spectrum of $V$ is the subset of $\hat\lieg$, defined as
    \[
        \sigma_\lieg(V)=\{ S\in \hat{\lieg} \mid \exists 
            k \colon \Tor^{U\lieg}_k(S^*,V)\neq 0\}.
    \]
\end{definition}
We will simply write $\sigma(V)$ instead of $\sigma_\lieg(V)$ if it is clear what Lie algebra is
considered. Till the and of the article \emph{the spectrum} stands for \emph{the Taylor spectrum}.

Let us provide several equivalent definitions, in order to use later.
\begin{lemma} \label{l:definitions}
   For an arbitary $S\in\lieg$, the following are equivalent:
   \begin{enumerate}
       \item $S \in \sigma(V)$; 
           
       \item $H_k(\lieg, V_{-S}) \ne 0$ for some $k$; 

       \item $H^k(\lieg, \bigwedge^n\lieg\otimes_\CC V_{-S}) \ne 0$ for some $k$;

       \item $S^* \otimes_\CC \bigwedge^n\lieg \in \sigma(V^*)$
   \end{enumerate}
\end{lemma}
\begin{proof}
    \begin{itemize}
        \item[$1 \Leftrightarrow 2$]
            This follows immediatly from Lemma \ref{t:torhomology}.

        \item[$2 \Leftrightarrow 3$]
            Poincar\'e duality (Theorem \ref{t:poincare}).

        \item[$3 \Leftrightarrow 4$]
            Again, by Poincar\'e duality we have
            \[
                H^k(\lieg, \bigwedge^n\lieg \otimes_\CC V_{-S})^* = H^{n - k} (\lieg, (V_{-S})^*).
            \]
            First observe that $(V_{-S})^*$ is isomorphic to $\bigwedge^n\lieg \otimes_{\CC}
            (V^*)_{-S^* \otimes_\CC \bigwedge^n\lieg}$. Using the equivalence of the 1'st and the
            3'rd statements we obtain the desired result. 
    \end{itemize}
\end{proof}
    In case of solvable Lie algebra $\lieg$ the set $\hat\lieg$, due to the Lie's Theorem, can be
    identified with the space of characters $(\lieg/[\lieg, \lieg])^*$. If $V$ is left
    $\lieg$-module, then the defined spectrum has the following relation, with the spectrum, given
    in (\ref{e:boassolarotonda}):
    \[
        \sigma_\lieg(V) = -\sigma^{BL}_\lieg(V^\circ),
    \]
    so we may think about $\sigma_\lieg$ as a generilization of $\sigma^{BL}_\lieg$.
