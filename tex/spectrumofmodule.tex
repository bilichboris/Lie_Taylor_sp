Let $\lieg$ be an arbitary Lie algebra and $V$ be a left $\lieg$-module. 
Recall, that $\hat\lieg$ is the set of isomorphism classes of simple
finite dimensional $\lieg$-modules.
\begin{definition}
    The Taylor spectrum of $V$ is the subset of $\hat\lieg$, defined as
    \[
        \sigma_\lieg(V)=\{ S\in \hat{\lieg} \mid \exists 
            k \colon \Tor^{U\lieg}_k(S^*,V)\neq 0\}.
    \]
\end{definition}
We will simply write $\sigma(V)$ instead of $\sigma_\lieg(V)$ if it is clear what Lie algebra is
considered.

Let us provide several equivalent definitions, in order to use later.
\begin{theorem}
   For an arbitary $S\in\lieg$, the following are equivalent:
   \begin{enumerate}
       \item $S \in \sigma(V)$; 
           
       \item $H_k(\lieg, V_{-S}) \ne 0$ for some $k$;

       \item $H^k(\lieg, \bigwedge^n\lieg\otimes_\CC V_{-S}) \ne 0$ for some $k$;

       \item $S^* \otimes_\CC \bigwedge^n\lieg \in \sigma(V^*)$
   \end{enumerate}
\end{theorem}
\begin{proof}
    \begin{itemize}
        \item[$1 \Rightarrow 2$]
            This follows immediatly from Lemma \ref{t:torhomology}.

        \item[$2 \Rightarrow 3$]
            Poincare duality (Theorem \ref{t:poincare}).

        \item[$3 \Rightarrow 4$] \todo{finish proof}

    \end{itemize}
\end{proof}
