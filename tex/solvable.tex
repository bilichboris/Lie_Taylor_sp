\subsection{The spectrum of the trivial module}
In this section $\lieg$ will denote an arbitary solvable Lie algebra of dimension $n$. Due to Lie's
theorem, every simple $\lieg$-module is one-dimensional, so we will identify $\hat\lieg$ with the
space of characters $(\lieg/[\lieg,\lieg])^*$. In this case, computing Taylor spectrum is a hard
job even for the trivial module $\CC$. For a finite-dimensional $\lieg$-module $V$ by the set of
weights $\omega(V) \subset \hat\lieg$ we mean the set of diagonal matrix entries in triangular
basis for $V$. It is independant on the choice of upper triangular basis and can be also described
as the set of one-dimensional subfactors of $V$. Consider the adjoint representation $\ad\lieg \in
\lmod\lieg$. The set of weights $\omega(\ad\lieg)$ is called Jordan-H{\"o}lder values of $\lieg$.
We denote by $2\rho$ the sum of all Jordan-H{\"o}lder values with multiplicites. It is exactly the
weight of the module $\bigwedge^n\lieg$, defined in section \ref{sec:preliminaries}.  The following
restriction on possible elements in the spectrum of the trivial module holds.
\begin{theorem} \label{t:jordanholder}
    For any solvable Lie algebra $\lieg$ of dimension $n$, if $\lambda \in \hat\lieg$ is in the
    spectrum $\sigma_\lieg(\CC)$, then it is the sum of at most $n$ Jordan-H{\"o}lder values of
    $\lieg$. Moreover, if $\lambda \in \sigma_\lieg(\CC)$, then $2\rho - \lambda$ is also in
    $\sigma_\lieg(\CC)$.
\end{theorem}
\begin{proof}
    The proof is by induction on $n$. Assertion of the Theorem, obviously, holds for
    one-dimensional Lie Algebra, so we have $n=1$ as the base of induction. Assume, that it holds
    for all solvable Lie algebras of dimension $n-1$. Choose any one-dimensional ideal
    $\mathfrak{c}$ in $\lieg$ and denote the corresponding character by $\mu$. By the third
    statement of the Theorem \ref{t:1ext} we know, that the spectrum $\sigma_\lieg(\CC)$ lies in
    the set $\{0, \mu\} + \sigma_{\lieg/\mathfrak{e}}(\CC)$. By induction, any $\nu \in
    \sigma_{\lieg/\mathfrak{e}}(\CC)$ is the sum of at most $n-1$ Jordan-H{\"o}lder values of
    $\lieg/\mathfrak{e}$, which are also Jordan-H{\"o}lder values of $\lieg$. This is the desired
    conclusion.

    For the second assertion we use Lemma \ref{l:definitions}. The trivial module is isomorphic to
    its dual, so if $\CC_\lambda$ is in the spectrum of $\CC$, then $\CC_\lambda^* \otimes
    \bigwedge^n\lieg \cong_\CC \CC_{2\rho - \lambda}$ is also in the spectrum.
\end{proof}
Obviously, $0$ is always in the spectrum and, hence, so is $2\rho$. So, generally, there are often
more than one element in the spectrum of $\CC$. Nevertheless, for some good classes of solvable Lie
algebras the situation is more clear, than in general case. One of such classes is nilpotent Lie
algebras.

\subsection{The spectrum of finite-dimensional modules over nilpotent Lie algebras}
The following well-known fact in representation theory for nilpotent Lie algebras plays major role
in this paragraph.
\begin{lemma} \label{l:nilpdecomp}
    For $V$ -- finite-dimensional indecomposable module over nilpotent Lie algebra $\lieg$,
    $\omega(V)$ consists of one element.
\end{lemma}
\begin{proof}
    Proposition 9 in chapter VII of \todo{ref bourbaki}
\end{proof}
We call modules with only one weight monoweighted. The last Lemma show us, that any
finite-dimensional module over nilpotent Lie algebra can be decomposed in sum of monoweighted
submodules.  Now we are ready to formulate the main result.
\begin{theorem}
    Let $\lieg$ be a nilpotent Lie algebra and $V$ -- finite-dimensional $\lieg$-module. Then the
    spectrum $\sigma(V)$ coincides with the set of weights $\omega(V)$.
\end{theorem}
\begin{proof}
    As in the proof of \ref{t:ssspectrumissubmodules} we may assume $V$ is monoweighted. The
    spectrum of module $V_{\lambda}$ is $\sigma(V) + \lambda$, so we may additionally assume, that
    $\omega(V) = \{0\}$. Observe, that by Engel's theorem, all Jordan-H{\"o}lder values of $\lieg$
    are zero, so by Theorem \ref{t:jordanholder}, we conclude that the assertion holds for trivial
    $V$. We now proceed by induction on $m = \dim V$. Choose some one-dimensional submodule in $V$.
    We have a short exact sequence
    \[
        0\to \CC \to V \to V/\CC \to 0.
    \]
    It follows easily from long exact sequence of cohomolohgies, that $\sigma(V) \subset
    \sigma(\CC) \cup \sigma(V/\CC)$ and the latest is equal $\{0\}$ by induction. On the other hand
    $0$ is in the spectrum of $V$, because $H_0(\lieg, V) = V_\lieg \neq 0$. This finishes the
    proof.
\end{proof}
