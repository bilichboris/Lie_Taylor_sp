\subsection{The spectrum of the trivial module}
In this section $\lieg$ will denote an arbitary solvable Lie algebra of dimension $n$. Due to Lie's
theorem, every simple $\lieg$-module is one-dimensional, so we will identify $\hat\lieg$ with the
space of characters $(\lieg/[\lieg,\lieg])^*$. In this case, computing Taylor spectrum is a hard
job even for the trivial module $\CC$. For a finite-dimensional $\lieg$-module $V$ by the set of
weights $\omega(V) \subset \hat\lieg$ we mean the set of diagonal matrix entries in triangular
basis for $V$. It is independent on the choice of upper triangular basis and can be also described
as the set of one-dimensional subfactors of $V$\cite{humphreys}. Consider the adjoint
representation $\ad\lieg \in \lmod\lieg$. The set of weights $\omega(\ad\lieg)$ is called
Jordan-H{\"o}lder values of $\lieg$.  We denote by $2\rho$ the sum of all Jordan-H{\"o}lder values
with multiplicites. By definition, it is the character of the adjoint representation, so it
coincides with  the weight of the module $\bigwedge^n\lieg$, defined in section
\ref{sec:preliminaries}, because, by definition, $2\rho$.  The following restriction on possible elements in the spectrum of the
trivial module holds.
\begin{theorem} \label{t:jordanholder}
    For any solvable Lie algebra $\lieg$ of dimension $n$, if $\lambda \in \hat\lieg$ is in the
    spectrum $\sigma_\lieg(\CC)$, then it is the sum of at most $n$ Jordan-H{\"o}lder values of
    $\lieg$. Moreover, if $\lambda \in \sigma_\lieg(\CC)$, then $2\rho - \lambda$ is also in
    $\sigma_\lieg(\CC)$.
\end{theorem}
\begin{proof}
    The proof is by induction on $n$. Assertion of the Theorem, obviously, holds for
    one-dimensional Lie Algebra, so we have $n=1$ as the base of induction. Assume, that it holds
    for all solvable Lie algebras of dimension $n-1$. Choose any one-dimensional ideal
    $\mathfrak{c}$ in $\lieg$ and denote the corresponding character by $\mu$. By the third
    statement of the Theorem \ref{t:1ext} we know, that the spectrum $\sigma_\lieg(\CC)$ lies in
    the set $\{0, \mu\} + \sigma_{\lieg/\mathfrak{e}}(\CC)$. By induction, any $\nu \in
    \sigma_{\lieg/\mathfrak{e}}(\CC)$ is the sum of at most $n-1$ Jordan-H{\"o}lder values of
    $\lieg/\mathfrak{e}$, which are also Jordan-H{\"o}lder values of $\lieg$. This is the desired
    conclusion.

    For the second assertion we use Lemma \ref{l:definitions}. The trivial module is isomorphic to
    its dual, so if $\CC_\lambda$ is in the spectrum of $\CC$, then $\CC_\lambda^* \otimes
    \bigwedge^n\lieg \cong_\CC \CC_{2\rho - \lambda}$ is also in the spectrum.
\end{proof}
The first part of the theorem was originally obtained by S. Wadsley in the converstion on the
mathoverflow website.  

Obviously, $0$ is always in the spectrum and, hence, so is $2\rho$. So,
generally, there is often more than one element in the spectrum of $\CC$. 
\begin{example}
   Let $\lieg$ be a 3-dimensional solvable Lie algebra with basis $e_1, e_2, e_3$ and the
   commutator, given by $[e_1, e_2] = e_2$ and $[e_1, e_3] = \lambda e_3$ for some $\lambda \in
   \CC$. The space of characters is one dimensional, so we identify it with $\CC$ by evaluation on
   $e_1$. Then $\sigma_\lieg(\CC) = \{0,1,\lambda, 1 + \lambda\}$. Indeed, $0$ and $2\rho = 1 +
   \lambda$ is always in the spectrum, and for $1$ and $\lambda$ the first homology groups are
   non-vanishing.
\end{example}

Nevertheless, for some nice classes of solvable Lie
algebras the situation is more clear, than in general case. One of such classes is nilpotent Lie
algebras.

\subsection{The spectrum of finite-dimensional modules over nilpotent Lie algebras}
The following well-known fact in representation theory for nilpotent Lie algebras plays major role
in this paragraph.
\begin{lemma} \label{l:nilpdecomp}
    For $V$ -- a finite-dimensional indecomposable module over nilpotent Lie algebra $\lieg$,
    $\omega(V)$ consists of one element.
\end{lemma}
\begin{proof}
    Proposition 9 in chapter VII of \cite{bourbaki}.
\end{proof}
We call modules with only one weight monoweighted. The last Lemma show us, that any
finite-dimensional module over nilpotent Lie algebra can be decomposed in the sum of monoweighted
submodules.  Now we are ready to formulate the main result.
\begin{theorem} \label{t:nilpspectrum}
    Let $\lieg$ be a nilpotent Lie algebra and $V$ -- finite-dimensional $\lieg$-module. Then the
    spectrum $\sigma(V)$ coincides with the set of weights $\omega(V)$, that is, with the set of
    all one-dimensional submodules.
\end{theorem}
\begin{proof}
    As in the proof of Theorem \ref{t:ssspectrumissubmodules} we may assume $V$ is monoweighted. The
    spectrum of module $V_{\lambda}$ is $\sigma(V) + \lambda$, so we may additionally assume, that
    $\omega(V) = \{0\}$. Observe, that by Engel's theorem, all Jordan-H{\"o}lder values of $\lieg$
    are zero, so by Theorem \ref{t:jordanholder}, we conclude that the assertion holds for trivial
    $V$. We now proceed by induction on $m = \dim V$. Choose some one-dimensional submodule in $V$.
    We have a short exact sequence
    \[
        0\to \CC \to V \to V/\CC \to 0.
    \]
    It follows easily from long exact sequence of cohomologies, that $\sigma(V) \subset
    \sigma(\CC) \cup \sigma(V/\CC)$ and the latest is equal $\{0\}$ by induction. On the other hand
    $0$ is in the spectrum of $V$, because $H_0(\lieg, V) = V_\lieg \neq 0$. This finishes the
    proof.
\end{proof}

\subsection{Case of borel subalgebras of semisimple Lie algebras}
%Пусть $\lies$ -- полупростая алгебра Ли и $\lieg$ -- некоторая её борелевская подалгебра. Известно, что $\lieg = \lieh \ltimes \lien$, где $\lieh$ -- картановская подалгебра $\lies$, а $\lien = [\lieg, \lieg]$. Введем обозначение $\Delta, \Delta^+ \subset \lieh^* = \lieg^*$ для системы корней относительно $\lieh$ и множества положительных корней относительно $\lieg$ соответственно. В терминах спектра Тейлора $\Delta = \sigma_h(\lies)$ и $\Delta^+ = \sigma_h(\lien)$ (так как $\lieh$ -- абелева, то это просто множество совместных собственных значений). За $W$ обозначим группу Вейля относительно $\Delta^+$. Для каждого элемента $w\in W$ определена его длина $0 \leq l(w) \leq \dim \lien$

%Нашей задачей является описание спектра Тейлора тривиального модуля алгебры Ли $\lieg$. Заметим, что эта задача эквивалентна следующей: для каких одномерных $\lieg$-модулей $\cmpl_\lambda$ его гомологии $H^{\mathrm{Lie}}_{\bullet}(\lieg, \cmpl_\lambda)$ не зануляются? Далее мы приведем полное решение этой задачи, более того мы получим явный вид соответсвующих гомологий. 
Let $\lies$ be a semisimple Lie algeba and $\lieg$ its Borel subalgebra. It is known
\cite{humphreys}, that $\lieg$ is isomorphic to semidirect product $\lieh \ltimes \lien$, where
$\lieh$ is a Cartan subalgebra of $\lies$ and $\lien = [\lieg, \lieg]$. By $\Delta \subset \lieh^*
= \hat\lieg$ we will denote the root system of $\lies$ relative to $\lieh$ and $\lieg$. We write
$\Delta^+ \subset \Delta$ for denote the subset of positive roots. In fact, elements of $\Delta^+$
are nonzero Jordan-H\"older values of $\lieg$. Let $W(\Delta)$ denote the Weyl group. For any
element $w\in W$ its length is denoted by $l(w)$. 

The aim of the paragraph is to describe the spectrum of irreducible $\lies$ modules, regarded
as $\lieg$ modules. The plan is to compute relveant cohomologies of $\lien$ and then take advantage
of the Hochshild-Serre spectral sequence. Fortunately, these two steps were alredy done by
different authors and we only need to glue them together. The first one is Kostant Theorem.
\begin{theorem}[Kostant]
   Let $V$ be irreducible representation of semisiple Lie algebra $\lies$ with highest weight
   $\lambda$, relative to borel subalgebra $\lieg = \lieh \ltimes \lien$. Then, as $\lieh$-module,
   $H^k(\lien, V)$ is the sum of one-dimensional modules of weights
   \[
       w(\lambda + \rho) - \rho,~\in W(\Delta), ~l(w) = k,
   \]
   where $\rho$ is half-sum of positive roots (or, equivalently, Jordan-H\"older values).
\end{theorem}
\begin{proof}
    Theorem 6.12 in \cite{knapp}.
\end{proof}
We say, that an abelian Lie algebra $\lieh$ acts torally on $\lieh$-module $W$ if $W$ is direct sum
of one-dimensional submodules. When $\lieh$ is a Cartan subalgebra of semisiple Lie algebra
$\lies$, $\lieh$ acts torally on any finite dimensional $\lies$-module. If $\lieg = \lieh \ltimes
\lien$ -- the Borel subalgebra of $\lies$, then $\lieh$ also acts torally on $\lien$. The following
Theorem guarantees convergence of the Hochsild-Serre spectral sequence in the special case of
semidirect products.
\begin{theorem}
    Let $\lieg=\lieh\ltimes\lien$ be a semidirect product where $\lieh$ is abelian and acts torally
    both on $\lieg$ and a $\lieg$-module $V$. Then with respect to the weighting induced by the
    action of $\lieh$ we have 
    \begin{equation} \label{t:coll:1}
        H^k(\lieg, V) = \bigoplus_{p + q = k} \bigwedge^p\lieh^* \otimes_\CC H^q(\lien, V)^\lieh
    \end{equation} 
\end{theorem}
\begin{proof}
    Theorem 4 in \cite{coll}. 
\end{proof}

Let $\mu \in \lieh^*$ be a weight. For any $\lieg$-module $V$, one may observe, that $H^k(\lien,
V_{-\mu}) \cong H^k(\lien, V)_{-\mu}$ for all $k$. Hence, if $V$ is irreducible $\lies$-module of
highest weight $\lambda$, then $H^q(\lien, V_{-\mu})^\lieh$ from (\ref{t:coll:1}) is nonzero for
some $q$ if and only if $\mu = w(\lambda + \rho) - \rho$. Combining it with Lemma
$\ref{l:definitions}$ we obtain the main result.
\begin{theorem} \label{t:borelspectrum}
   For an irreducible module $V$ of highest weight $\lambda$ over the semisimple Lie algebra
   $\lies$, the Taylor spectrum of $V$, regarded as the module over Borel subalgebra $\lieg \subset
   \lies$ has the form 
   \[
       \sigma_\lieg(V) = \{ \rho + w(\lambda + \rho)\colon w \in W(\Delta)\}
   \]
\end{theorem}
\begin{proof}
    Indeed, by Lemma \ref{l:definitions}, $\nu \in \sigma_{\lieg}(V)$ if and only if $H^k(\lieg,
    \bigwedge^n\lieg \otimes_\CC V_{-\nu})$ are nonzero for some $k$. As stated in the begining of
    the section, $\bigwedge^n \lieg$ is isomorphic to $\CC_{2\rho}$, so $\bigwedge^n\lieg
    \otimes_\CC V_{-\nu} \cong V_{2\rho - \nu}$. As shown above, the cohomology groups $H^k(\lieg,
    V_{2\rho - \nu})$ are nonzero exactly when $2\rho - \nu = \rho - w(\lambda + \rho)$.
\end{proof}
